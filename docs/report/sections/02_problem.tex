\chapter{Problem Description}

The renewed interest in commercial supersonic air transport has motivated the preliminary design of propulsion systems capable of efficient operation at high supersonic cruise conditions. In this study, a conceptual commercial aircraft designed to cruise at Mach 3.2 is considered, with the objective of completing a preliminary comparison between two propulsion system candidates. The primary emphasis of the project is the design and analysis of the propulsion system rather than full aircraft sizing or configuration.

A critical component of supersonic propulsion system performance is the inlet, whose function is to decelerate the freestream supersonic flow to subsonic conditions suitable for the engine while minimizing total pressure losses. Poor inlet pressure recovery directly reduces available thrust and overall cycle efficiency. As such, the inlet design strongly influences the feasibility of high-speed commercial propulsion systems.

The first phase of this project focuses on the preliminary design of a supersonic inlet for a fictional engine using classical compressible flow theory. The inlet is designed for a cruise Mach number of 3.2 and employs a multi-shock compression system consisting of three oblique shocks followed by a terminal normal shock. The inlet is optimized using the Oswatitsch equal-strength shock principle to maximize total pressure recovery across the inlet.