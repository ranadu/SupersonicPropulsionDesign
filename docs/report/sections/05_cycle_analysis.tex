\chapter{Parametric Cycle Analysis}

A one-dimensional parametric cycle analysis was performed to evaluate the performance of a fictional turbojet engine designed specifically for sustained Mach 3.2 cruise. The analysis incorporated the inlet pressure recovery obtained from the Oswatitsch-optimized inlet design presented in Part I, with a total inlet pressure ratio of $\pi_d = 0.793$.

The cycle analysis examined the influence of compressor pressure ratio ($\pi_c$) and turbine inlet temperature ($T_{t4}$) on key performance metrics, including specific thrust and thrust-specific fuel consumption (TSFC). The flight condition was fixed at Mach 3.2 and an altitude of 15 km to represent the intended cruise condition.

\section{Parametric Study Setup}

The compressor pressure ratio was varied from $\pi_c = 8$ to $30$, while turbine inlet temperature was varied from $T_{t4} = 1500$ K to $2000$ K. Component efficiencies and pressure losses were selected based on typical values for preliminary turbojet analysis. Regions producing negative or non-physical thrust were excluded from the TSFC evaluation to ensure physically meaningful results.

\section{Parametric Trends}

The parametric results demonstrate that specific thrust decreases with increasing compressor pressure ratio at high supersonic flight speeds. This trend reflects the increasing compressor work requirement and diminishing benefit of higher pressure ratios at elevated inlet total temperatures. Conversely, specific thrust increases strongly with turbine inlet temperature, indicating that high $T_{t4}$ values are required to overcome ram drag at Mach 3.2.

The TSFC results exhibit a trade-off between efficiency and thrust. Low $T_{t4}$ and low $\pi_c$ configurations minimize TSFC but fail to provide sufficient specific thrust for sustained cruise. Imposing a minimum specific thrust constraint shifts the optimal design toward higher turbine inlet temperatures and moderate compressor pressure ratios.