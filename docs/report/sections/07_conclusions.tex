% \chapter{Conclusions}

% A preliminary supersonic inlet design for a Mach 3.2 commercial transport application was completed using the Oswatitsch equal-strength shock principle. A three-oblique-shock compression system followed by a terminal normal shock was successfully designed to meet the specified inlet requirements.

% The resulting inlet achieves a total pressure recovery of 0.793, which is consistent with classical expectations for high-supersonic external compression inlets. The smooth Mach number reduction and absence of detached shocks indicate a physically realistic and robust inlet configuration suitable for preliminary propulsion system design.

% The optimized inlet parameters obtained in this phase form the basis for the parametric cycle analysis and propulsion system trade study conducted in Part II of this project.

\chapter{Conclusions}

A preliminary propulsion system design and trade study was conducted for a commercial aircraft intended to cruise at Mach 3.2. An Oswatitsch-optimized supersonic inlet was successfully designed to maximize total pressure recovery, yielding an inlet pressure ratio of 0.793. This inlet formed the foundation for subsequent cycle analysis.

A parametric one-dimensional cycle analysis demonstrated that sustained Mach 3.2 cruise requires elevated turbine inlet temperatures and moderate compressor pressure ratios to maintain positive net thrust. A fictional turbojet engine designed specifically for the mission achieved a specific thrust of approximately 390 N/(kg/s) at a turbine inlet temperature of 2000 K.

Comparison with the Olympus 593 off-the-shelf engine revealed that, while the Olympus was well-suited for Mach 2 cruise, it could not sustain Mach 3.2 operation in its baseline configuration. This outcome emphasizes the importance of tailoring propulsion system design to specific mission requirements rather than extrapolating existing designs beyond their intended operating envelopes.

Overall, the study demonstrates that achieving commercial Mach 3.2 cruise necessitates significant advancements in turbine materials, cooling technology, and inlet performance, and underscores the value of mission-driven propulsion system design in preliminary aerospace engineering studies.